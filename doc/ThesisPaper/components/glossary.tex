% !TEX root = ../main.tex
\newglossaryentry{computer}
{
  name=computer,
  description={is a programmable machine that receives input,
               stores and manipulates data, and provides
               output in a useful format}
}

\newglossaryentry{poc}
{
  name={proof of concept},
  description={}
  }
\newglossaryentry{ui}
{
  name={user interface},
  description={}
  }
\newglossaryentry{ai}
{
  name={arithmetic intensity},
  description={a measure of floating-point operations (FLOPs)
              \hyphenation{per-formed} performed by a \hyphenation{gi-ven} given code or code section relative
              to the amount of memory accesses (Bytes) that are required
               to support those operations\cite{AI}}
  }

\newglossaryentry{speed-up}
{
  name={speed-up},
  description={the factor of temporal acceleration a program
  exhibits when additional computational resources are dedicated to it's execution.}
}

\newglossaryentry{directive pragmas}
{
  name={directive pragma},
  description={a computer programming language construct that specifies how a compiler
  should process input data} % sourced from wikipedia
}


\newglossaryentry{rc}{%SOURCE: wikipedia
name={race condition},
description={A race condition or race hazard is the behavior of an electronic,
 software, or other system where the output is dependent on the sequence or
 timing of other uncontrollable events. It becomes a bug when events do not
 happen in the order the programmer intended. The term originates with the idea
 of two signals racing each other to influence the output first.}
}
\newglossaryentry{dd}{
name={data dependencies},
description={}
}
\newglossaryentry{sisd}{
name={single instruction single data},
description={}
}
\newglossaryentry{simt}{
name={single instruction multiple threads},
description={}
}

\newglossaryentry{simd}{
name={single instruction multiple data},
description={}
}
\newglossaryentry{gp}{%SOURCE: wikipedia
name={Gaussian Plane},
description={The two dimensional plane of complex numbers.}
}
\newglossaryentry{CURAND}{
name={CURAND},
description={
The CURAND library provides facilities that focus on the simple and efficient
generation of high-quality pseudorandom and quasirandom numbers.\cite{cuRAND}
}
}

\newacronym[longplural={partial differential equations}]{PDE}{PDE}{partial differential equations}
\newacronym{mpi}{MPI}{Message Passing Interface}

\newacronym[longplural={Random Walks on Spheres}]{RWoS}{RWoS}{Random Walk on Spheres}

\newacronym[longplural={graphical processing units}]{GPU}{GPU}{graphical processing unit}

\newacronym[longplural={central processing units}]{CPU}{CPU}{central processing unit}
\newacronym{hpc}{HPC}{high performance computing}

\newacronym[longplural={arithmetic logic units}]{ALU}{ALU}{arithmetic logic unit}

\newacronym[longplural={streaming multi-processors}]{SM}{SM}{streaming multi-processor}

\newacronym[longplural={boundary value problems}]{BVP}{BVP}{boundary value problem}
\newacronym[longplural={general purpose graphical processing units}]{GPGPU}{GPGPU}{general purpose graphical processing units}
\newacronym{CUDA}{CUDA}{compute unified device architecture}
\newacronym{RAM}{RAM}{random access memory}
\newacronym{SRAM}{SRAM}{static random access memory}
\newacronym{DRAM}{DRAM}{dynamic random access memory}
\newacronym{I/O}{I/O}{input/output}
\newacronym{PTX}{PTX}{Parallel Thread eXecution}
\newacronym{jit}{JIT}{just in time}
