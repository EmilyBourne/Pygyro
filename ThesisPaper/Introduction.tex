\rchapter{Introduction}

As an interpreted language, Python runs significantly slower than other languages such as C or Fortran. However code can usually be developed much faster in Python. As a result it is preferred by many developers for smaller projects. Although the resulting program runs slower, the trade-off is acceptable as the overall time spent on the project is lower.

The situation is more problematic for large projects. In this case the runtime can be several days. A program written in python can run as much as 100 times slower than its equivalent in C or Fortran. This would therefore lead to a runtime of close to one year. This is especially problematic as the simplicity of python means that many scientists rarely use lower level languages and are thus out of practice when they need to use them.

Multiple solutions exist to try and improve this situation which consist of accelerating python code so that it runs at speeds which more closely approach those seen in C or Fortran. Just-in-time compilers such as PyPy and 